\documentclass[11pt,letterpaper]{article}
\usepackage{cogsys}
\usepackage{cogsysapa}
% \usepackage{apacite}
% \usepackage{graphicx}
\usepackage[T1]{fontenc}
\usepackage{times}
\usepackage[pdftex]{graphicx} % use this when importing PDF files
% \usepackage{epsf} 

\cogsysheading{1}{2012}{1-6}{11/2011}{8/2012}

\ShortHeadings{Formatting Instructions}
              {P.\ Langley, G.\ Hunt, and D.\ G.\ Shapiro}

\begin{document} 

\title{Formatting Instructions for Advances in Cognitive Systems}
 
\author{Pat Langley}{langley@asu.edu}
\author{Glen Hunt}{glen.hunt@asu.edu}
\address{Computing Science and Engineering, Arizona State University, 
         Tempe, AZ 85287 USA}
\author{Daniel G.\ Shapiro}{dgs@isle.org}
\address{Institute for the Study of Learning and Expertise, 
         2164 Staunton Court, Palo Alto, CA 94306 USA}
\vskip 0.2in
 
\begin{abstract}Keep your abstract brief, limiting it to one paragraph 
and to fewer than ten sentences. The abstract should begin 0.35~inches 
(0.9~cm) below the final address. The heading `Abstract' should be
centered, bold, and in 12~point type. The abstract body should use
10~point type and should be indented 0.25~inches (0.635~cm) more than
normal on left-hand and right-hand margins. Insert about 0.3~inches 
(0.762~cm) of blank space after the body.
\end{abstract}

\section{Format of the Paper} 
 
All submissions should follow the same format to ensure that the 
printers can reproduce them without problems and to let readers 
more easily find the information that they desire.

\subsection{Paper Dimensions}

The text of the paper should be formatted in one column, with an
overall width of 6.0~inches (15.24~cm) and length of 8.0~inches 
(20.32~cm). The left margin should be 1.25~inches (3.175~cm) and the 
top margin 1.5~inches (3.81~cm). The right and bottom margins will
depend on whether you print on US letter or A4 paper.

The paper body should be set in 11~point type with a vertical spacing 
of 12~points. Please use Times Roman typeface throughout the text. 

\subsection{Title and Author Information}

The paper title should be set in 14~point bold type and centered
between two horizontal rules that are 1~point thick, with 1.5~inches
between the top rule and the top edge of the page. Capitalize the
first letter of each content word and put the rest of the title in 
lower case.

Author information should start 0.3~inches (0.762~cm) below the bottom
rule surrounding the title. The authors' names should appear in
11~point bold type, electronic mail addresses in 11~point small
capitals, and physical addresses in ordinary 11~point type.

Each author's name should be flush left, whereas the email address
should be flush right on the same line. The author's physical address
should appear flush left on the ensuing line, on a single line if
possible. If successive authors have the same affiliation and address, 
then give this information only once for them. 

\subsection{Partitioning the Text} 

You should organize your paper into sections and paragraphs to help 
readers place a structure on the material and understand its 
contributions. 

% Use \vspace for fine control of spacing above and below headings. 
\vspace{-0.018in}
\subsubsection{Sections and Subsections}
\vspace{-0.015in}

Section headings should be numbered, flush left, and set in 12~point 
bold type with the content words capitalized. Leave about 0.25~inches
(0.635~cm) of space before the heading and approximately 0.15~inches 
(0.33~cm) after the heading.

Similarly, subsection headings should be numbered, flush left, and 
set in 11~point bold type with the content words capitalized. Leave
approximately 0.2~inches (0.51~cm) of space before the heading and 
0.13~inches (0.33~cm) afterward.

Finally, subsubsection headings should be numbered, flush left, and
set in 11~point italics with the content words beginning with capitals. 
Leave about 0.18~inches (0.457~cm) of space before the heading and
about 0.1~inches (0.254~cm) after the heading. Please use no more 
than three levels of headings in a paper.

\subsubsection{Paragraphs, Quotations, and Footnotes}

Within each section or subsection, you should further partition 
the paper into paragraphs. You should not indent the first line 
of a section or subsection's initial paragraph, but you should
indent successive paragraphs by 0.25 inches (0.635~cm). There 
should be no extra spacing between paragraphs within a section. 
 
You may want to quote an excerpt of relevant text that appears in 
the published work of some other authors: 

\begin{quote}
Such a quotation should begin 0.25 inches (0.635 cm) below the preceding 
line of regular text and end 0.25 inches (0.635 cm) before it begins
again. The left and right margins should be 0.35 inches (0.889 cm) larger 
than for the regular text, but the quoted text should have the same
font as regular text, with a vertical spacing of 12~points.
\end{quote}

\noindent
You should remember to cite the source of the quotation, and you should 
close off each quotation with some regular text, which should not be
indented.

You can use footnotes\footnote{For the sake of readability, footnotes
should be complete sentences.} to provide readers with additional
information about a topic without interrupting the flow of the paper.
Indicate footnotes with a number in the text where the point is most
relevant. Place the footnote in 10~point type at the bottom of the
page on which it appears. Precede the first footnote on a page 
with a horizontal rule of 2.0~inches (5.08~cm) to separate it from 
the main text.\footnote{Multiple footnotes can appear on each page, 
in the same order as they appear in the text, but spread them across
pages if possible.}

\subsubsection{Itemized and Enumerated Lists}

When you have a number of related but distinct points, you may want
to organize them in an itemized list that: 

\vskip 0.05in

\cbullet 
begins 0.15 inches (0.381 cm) below the preceding line of regular text 
and ends 0.15 inches (0.381 cm) before the regular text begins; 

\cbullet 
uses a left margin for each item that is 0.25 inches (0.635 cm) to the 
right of the left margin for regular text; 

\cbullet 
includes a bullet at the beginning of the first line of each item is
separated from the item's text by 0.075 inches (0.19 cm); and

\cbullet 
uses a vertical spacing of 12~points for lines within each items 
and separates successive items by the same amount. 

\vskip 0.05in
\noindent
In general, you should close off each itemized list with some sentences 
of regular text, which should not be indented. In some cases, you may
instead want to use an enumerated list that replaces bullets with numbers. 

\begin{figure}[t]
\vskip 0.05in
\begin{center}
% \setlength{\epsfxsize}{3.5in}
% \centerline{\epsfbox{cascade.eps}}
\includegraphics[width=3.5in]{cascade.eps}
% \vskip 0.1in
\caption{Modules in the {\sc Icarus} cognitive architecture and their
         cascaded organization.} 
\label{sample-figure}
\end{center}
\vskip -0.2in
\end{figure} 

\subsection{Figures}
 
You may want to include figures in the paper to help readers visualize
your approach and your results. Such artwork should be centered,
legible, and separated from the text. Lines should be dark and at
least 0.5~points thick for clear display and printing, and the text 
should not appear on a dark background.

% Use \newpage to insert a page break between paragraphs. 

% \newpage

Label all distinct components of each figure. If the figure takes
the form of a graph, then give a name for each axis and include a 
legend that briefly describes each curve. However, do {\it not\/} 
include a title above the figure, as the caption already serves
this function. 

Number figures sequentially, placing the figure number and caption
{\it after\/} the graphics, with at least 0.2~inches (0.508~cm) of 
space before the caption and 0.3~inches (0.762~cm) after it, as in
Figure~\ref{sample-figure}. The figure caption should be set in
10~point type and centered unless it runs two or more lines, in which
case it should be flush left. You may float a figure to the top or
bottom of a page, but ideally it should appear on the page that
first mentions it or on the page immediately afterward. 
 
\subsection{Tables} 
 
You may also want to include tables that summarize material. Like 
figures, these should be centered, legible, and numbered consecutively. 
However, place the title {\it above\/} the table with at least 
0.1~inches (0.254~cm) of space before the title and the same after 
it, as in Table~\ref{sample-table}. The table title should be set in
11~point type and centered unless it runs two or more lines, in which
case it should be flush left.

Tables contain textual material that can be typeset, as contrasted 
with figures, which contain graphical material that must be drawn. 
Specify the contents of each column in the table's topmost row. Again,
you may float a table to the top or bottom of a page, but attempt to 
place each table on the page that first mentions it or on the one
immediately afterward. 

% Note use of \abovespace and \belowspace to get reasonable spacing 
% above and below tabular lines. 

\def\w{$\oplus$}
\def\b{$\ominus$}
\def\h{$\odot$}

\begin{table}[t]
\vskip -0.15in
\caption{Adequacy of four models of learning in problem solving in 
         terms of whether they account (\w), fail to account (\b), 
         or partially account (\h) for phenomena from \cite{in-book-vanlehn}
         [{\tt *}] and Jones (1989) [$\diamond$].}
\label{sample-table}
\begin{small}
\begin{center}
% \vskip -0.10in
\begin{tabular}{lcccc}
\hline
\abovespace\belowspace
                       & ~~~ACT-R~~~ & ~~~Soar~~~ & {\sc ~~Eureka~~} & {\sc Daedalus} \\
\hline\abovespace
Means-ends analysis$^{\diamond}$  &   \h  &     \w    &      \w   &      \w \\
\hbox{Nonsystematicity$^{\diamond}$\hskip -0.2in} &   \b  &     \b    &      \w   &      \b \\
Problem isomorphs$^*$   &   \h  &     \h    &      \h   &      \h \\
Reduced search$^*$ &   \w  &     \w    &      \w   &      \w \\
Asymmetrical transfer$^*$    &   \w  &     \w    &      \w   &      \w \\
Einstellung effects$^*$    &   \w  &     \w    &      \w   &      \w \\
% Strategy differences   &   \h  &     \h    &      \h   &      \h \\
Reduced verbalization$^*$  &   \w  &     \w    &      \b   &      \b \\
Reduced solution time$^*$   &   \w  &     \w    &      \b   &      \b \\
% Self-monitoring        &   \h  &     \h    &      \b   &      \b \\
Rare analogies$^*$   &   \b  &     \b    &      \h   &      \h \\
\belowspace
% \belowstrut{0.10in}
Superficial analogies$^*$ &   \b  &     \b    &      \w   &      \w \\
\hline 
\end{tabular}
\end{center}
\vskip -0.10in
\end{small}
\end{table}
 
\subsection{Citations and References} 

Please use APA reference format regardless of your formatter or word
processor. If you rely on the \LaTeX\/ bibliographic facility, use the 
files {\tt\small cogsysapa.sty} and {\tt\small cogsysapa.bst}, which
are available on the Web site, to obtain this format. These require
you to place your references in a separate file with a {\tt\small bib}
extension, as explained in the \LaTeX\/ manual.

Citations within the text should include the authors' last names and
year. If the authors' names are included as part of the sentence, place 
only the year in parentheses, as in Jones and VanLehn (1994), but 
otherwise place the entire reference in parentheses with the authors
and year separated by a comma (Jones \& VanLehn, 1984). 
List multiple references alphabetically and separate them by semicolons 
\cite{paper-laird-etal,book-newell-simon}. Use the 
`et~al.'  construction only for citations with four or more authors or
after listing all authors to a publication in an earlier reference.

Use an unnumbered first-level section heading for the references, and 
use a hanging indent style, with the first line of the reference flush
against the left margin and subsequent lines indented by 10 points. 
The references at the end of this document include examples for journal
articles \singlecite{article-forbus}, conference publications \singlecite{paper-laird-etal},
book chapters \singlecite{in-book-vanlehn}, books \singlecite{book-newell-simon}, 
edited volumes \singlecite{book-shrager-langley}, technical reports 
\singlecite{techreport-shapiro-etal}, and dissertations \singlecite{phd-choi}.

Alphabetize references in both the text and at its end by the surnames
of the first authors, with single author entries preceding multiple
author entries. Order references for the same authors by year of
publication, with the earliest first.

\section{Electronic Formatting and Submission}

{\it Advances in Cognitive Systems\/} will rely on electronic submission 
of papers for review and publication. We assume that nearly all authors 
will have access to \LaTeX\/ or Word to format their documents and can
use a Web browser to download style files and upload their papers. 
Authors who do not have such access should send email with their
concerns to {\sl acs@cogsys.org}.

\subsection{Paper Length and Format}

The title {\it Advances in Cognitive Systems\/} is associated with
both an electronic journal and an annual conference, with many of the
articles appearing in the former having been submitted and reviewed
under the auspices of the latter. 

Papers submitted to the annual conference must not exceed sixteen (16)
pages, including all figures, tables, and references. We will return
to the authors any submissions that exceed this page limit or that
diverge significantly from the format specified herein.
Papers submitted for publication in the journal but not for presentation
at the meeting are not subject to this constraint. However, we encourage 
authors to keep their papers the same length as those for the conference.

Submissions may be accompanied by online appendices that contain data,
demonstrations, instructions for obtaining source code, or the source
code itself. We encourage authors to include such appendices when they
submit papers. This material will not count in a submission's page
length.

Electronic templates for producing the camera-ready copy are available
for \LaTeX\/ and Microsoft Word. Style files and sample papers are 
available on the Web at: 
\vskip 0.1in
\begin{small}
\centerline{{\tt http://www.cogsys.org/formats/} .}
\end{small}
\vskip 0.1in
\noindent
Authors who have questions about these electronic formats should direct
them to {\sl acs@cogsys.org}.

To ensure the ability to preview and print submissions, authors must
provide their manuscripts in pdf format. Papers prepared in Word
should be saved as pdf files and submitted in this format. To support
the review process, each submission must be accompanied by information
about the paper's title and abstract, as well as the authors' names
and physical addresses. 

\subsection{Submitting Initial and Final Papers}

Submission of initial and final papers will take place in a purely
electronic manner through the World Wide Web. Authors who intend to
submit a paper for presentation at the annual conference should upload
their file to the submission repository at
\vskip 0.1in
\begin{small}
\centerline{{\tt http://www.cogsys.org/conference/submit/} .}
\end{small}
\vskip 0.12in
\noindent
Submission should be completed no later that 11:59 PM Pacific time 
on the conference due date, which will be stated clearly on the 
meeting Web site. If a submission is late, then it will not be 
considered for inclusion at the meeting.

{\it Advances in Cognitive Systems\/} will also consider submissions
that are not intended for presentation at the annual conference. 
Authors who wish to submit papers of this sort should upload their
file to the submission repository at
\vskip 0.1in
\begin{small}
\centerline{{\tt http://www.cogsys.org/journal/submit/} .}
\end{small}
\vskip 0.08in
\noindent
Such submissions may occur at any time, unless they are related to 
special issues of the journal, which will have their own deadlines. 

% You are welcome to submit a paper simultaneously to another meeting 
% or publication, but only if you indicate this fact clearly on the
% submission form. Simultaneous submissions that are not clearly
% specified as such will be rejected on discovery. 

Final versions of papers accepted for publication, and resubmissions 
of conditionally accepted papers, should follow the same format 
as that for initial submissions. Authors should upload their revised 
papers to the same Web repository as used for the originals. 
For the annual conference, final versions should be deposited no 
later than 11:59 PM Pacific time on the due date. Authors of late 
submissions will not be provided with a slot to present their work 
at the meeting and their papers will not appear in the journal.
Papers not related to the conference may be uploaded at any time. 

In both cases, papers that have been conditionally accepted will be
examined for compliance with conditions placed on their acceptance. 
Final papers must follow the content, format and length restrictions 
that are specified in these instructions.

% \newpage
 
\begin{acknowledgements} 
\noindent
Please place your acknowledgements in an unnumbered section at the
end of the paper. Typically, this will include thanks to reviewers
who gave useful comments, to colleagues who contributed to the ideas, 
and to funding agencies or corporate sponsors that provided financial 
support.
\end{acknowledgements} 

\vspace{-0.25in}

{\parindent -10pt\leftskip 10pt\noindent
\bibliographystyle{cogsysapa}
\bibliography{format}

}

% Leave a blank line before the closing brace to ensure the final 
% reference has the proper indentation. 

\end{document} 

% Some authors prefer to typeset their references manually. In such
% cases, they can use the examples that follow as role models. 

\section*{References}

{\parindent -10pt\leftskip 10pt\noindent

% Journal article with multiple authors

% Cassimatis, N. L., Bello, P., \& Langley, P. (2008). Ability, parsimony
% and breadth in models of higher-order cognition. {\it Cognitive Science\/},
% {\it 33\/}, 1304--1322. 

% Doctoral dissertation

Choi, D.\ (2010). {\it Coordinated execution and goal management in a 
reactive cognitive architecture\/}. Doctoral dissertation, Department 
of Aeronautics and Astronautics, Stanford University, Stanford, CA.

% Journal article with single author

Forbus, K.\ D.\ (1984). Qualitative process theory. {\it Artificial
Intelligence\/}, {\it 24\/}, 85--168.

% Journal article with multiple authors

% Jones, R.~M., \& VanLehn, K. (1994). Acquisition of children's addition
% strategies: A model of impasse-free, knowledge-level learning. {\it
% Machine Learning\/}, {\it 16\/}, 11--36.

% Paper in conference proceedings with multiple authors

Laird, J.\ E., Rosenbloom, P.\ S., \& Newell, A.\ (1984). Towards 
chunking as a general learning mechanism. {\it Proceedings of the
Fourth National Conference on Artificial Intelligence\/} (pp.\ 
188--192). Austin, TX: Morgan Kaufmann.

% Book with two authors

Newell,~A., \& Simon,~H.~A. (1972). {\it Human problem solving\/}.
Englewood Cliffs, NJ: Prentice-Hall. 

% Unpublished technical report with multiple authors 

Shapiro, D., Billman, D., Marker, M., \& Langley, P. (2004). {\it A
human-centered approach to monitoring complex dynamic systems\/}
(Technical Report). Institute for the Study of Learning and Expertise,
Palo Alto, CA.

% Chapter in edited book with multiple editors

% Shrager, J., \& Langley, P. (1990). Computational approaches to
% scientific discovery. In J. Shrager \& P. Langley (Eds.), {\it
% Computational models of scientific discovery and theory formation}.
% San Mateo, CA: Morgan Kaufmann.

% Edited book with multiple editors

Shrager, J., \& Langley, P. (Eds.) (1990). {\it Computational models
of scientific discovery and theory formation}. San Mateo, CA: Morgan
Kaufmann.

% Chapter in edited book with single author and single editor

VanLehn, K. (1989). Problem solving and cognitive skill acquisition.
In M.\ I.\ Posner (Ed.), {\it Foundations of cognitive science\/}.
Cambridge, MA: MIT Press.

}
% Leave a blank line before the closing brace to ensure the final 
% reference has the proper indentation. 
